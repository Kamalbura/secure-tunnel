% ============================================================================
%  GLOSSARY
% ============================================================================
\chapter*{Glossary}
\addcontentsline{toc}{chapter}{Glossary}

\begin{description}[style=nextline,leftmargin=4cm,labelwidth=3.8cm]

\item[AAD] \textbf{Additional Authenticated Data.} Data that is integrity-protected but not encrypted. In this system, the packet header is bound as AAD so that tampering with the header causes decryption to fail.

\item[AEAD] \textbf{Authenticated Encryption with Associated Data.} A cryptographic construction that simultaneously provides confidentiality (encryption) and integrity (authentication). Examples used in this system: AES-256-GCM, ChaCha20-Poly1305, ASCON-128a.

\item[AES] \textbf{Advanced Encryption Standard.} A symmetric block cipher standardized by NIST in 2001. This system uses AES-256-GCM (256-bit key, Galois/Counter Mode).

\item[ASCON] A lightweight authenticated cipher designed for constrained environments. Winner of the NIST Lightweight Cryptography competition. This system supports the ASCON-128a variant.

\item[C2] \textbf{Command and Control.} The communication link between a ground station and a drone that carries commands (arm, takeoff, navigate) and telemetry (position, battery, attitude).

\item[ChaCha20-Poly1305] An AEAD cipher combining the ChaCha20 stream cipher with the Poly1305 authenticator. An alternative to AES-GCM that performs well without hardware AES support.

\item[Classic McEliece] A code-based post-quantum KEM with very large public keys but very small ciphertexts and fast operations. Based on Niederreiter's code-based cryptosystem from 1986.

\item[DVFS] \textbf{Dynamic Voltage and Frequency Scaling.} A power management technique where the CPU clock speed and voltage are adjusted dynamically. Relevant to energy measurements on the Raspberry Pi.

\item[Epoch] In this system, a one-byte counter (0--255) that is incremented on each rekey event. Combined with the sequence number to form a unique nonce for each packet.

\item[Falcon] A lattice-based digital signature algorithm based on NTRU lattices and fast Fourier sampling. Produces compact signatures but requires careful floating-point implementation.

\item[FC] \textbf{Flight Controller.} The embedded computer (typically a Pixhawk) that directly controls the drone's motors, reads sensors, and executes flight plans.

\item[FrodoKEM] A lattice-based KEM based on the Learning With Errors (LWE) problem. More conservative than ML-KEM but with larger key sizes.

\item[GCM] \textbf{Galois/Counter Mode.} An authenticated encryption mode for block ciphers. Provides both confidentiality and integrity with hardware acceleration on modern CPUs (via AES-NI).

\item[GCS] \textbf{Ground Control Station.} The operator's computer that displays telemetry, sends commands, and monitors the drone's status. In this system, a Windows PC running Mission Planner or QGroundControl.

\item[HKDF] \textbf{HMAC-based Key Derivation Function.} Used in this system to derive two directional transport keys from a single shared secret established during the handshake.

\item[HMAC] \textbf{Hash-based Message Authentication Code.} A construction for computing a keyed hash for authentication. Used in this system for drone PSK authentication during handshake.

\item[HQC] \textbf{Hamming Quasi-Cyclic.} A code-based post-quantum KEM. An alternative to lattice-based schemes.

\item[INA219] A Texas Instruments current/voltage monitor IC used via I2C to measure power consumption of the Raspberry Pi in real time.

\item[IV] \textbf{Initialization Vector.} A value used to ensure that encrypting the same plaintext twice produces different ciphertext. In this system, the IV is deterministic (derived from epoch and sequence number) and not transmitted on the wire.

\item[KDF] \textbf{Key Derivation Function.} A function that derives one or more cryptographic keys from a shared secret. This system uses HKDF-SHA256.

\item[KEM] \textbf{Key Encapsulation Mechanism.} A public-key primitive where one party generates a public key, the other party ``encapsulates'' a random shared secret using that public key, and the first party ``decapsulates'' using their secret key. Replaces traditional key exchange (like Diffie-Hellman).

\item[Lattice] A regular grid of points in multi-dimensional space. The mathematical structure underlying ML-KEM and ML-DSA. The hardness of finding short vectors in lattices (the Shortest Vector Problem) provides security.

\item[liboqs] \textbf{Open Quantum Safe library.} A C library implementing post-quantum cryptographic algorithms. This system uses its Python bindings (\texttt{oqs-python}).

\item[MAVLink] \textbf{Micro Air Vehicle Link.} A lightweight binary protocol for communication between drones and ground stations. Supports messages for telemetry, commands, parameters, and mission plans.

\item[MAVProxy] An open-source MAVLink proxy and ground station written in Python. Used in this system to bridge serial Pixhawk data to UDP and to provide GCS-side MAVLink routing.

\item[mDNS] \textbf{Multicast DNS.} A protocol for resolving hostnames on local networks without a central DNS server. Used optionally in this system for zero-configuration drone/GCS discovery.

\item[ML-DSA] \textbf{Module Lattice Digital Signature Algorithm.} The NIST-standardized post-quantum signature scheme (FIPS~204), formerly known as CRYSTALS-Dilithium. Available at security levels 2 (ML-DSA-44), 3 (ML-DSA-65), and 5 (ML-DSA-87).

\item[ML-KEM] \textbf{Module Lattice Key Encapsulation Mechanism.} The NIST-standardized post-quantum KEM (FIPS~203), formerly known as CRYSTALS-Kyber. Available at security levels 1 (ML-KEM-512), 3 (ML-KEM-768), and 5 (ML-KEM-1024).

\item[NIST] \textbf{National Institute of Standards and Technology.} The U.S.\ federal agency responsible for cryptographic standards. Their post-quantum cryptography standardization process (2017--2024) produced the algorithms used in this system.

\item[NIST Security Level] A classification of post-quantum algorithm strength: Level~1 ($\approx$AES-128), Level~3 ($\approx$AES-192), Level~5 ($\approx$AES-256).

\item[Nonce] \textbf{Number used once.} A value that must never be reused with the same key. In this system, constructed deterministically from the epoch byte and a monotonic sequence number.

\item[Pixhawk] An open-source flight controller hardware platform running ArduPilot or PX4 firmware. The drone in this system uses a Pixhawk connected via USB serial to a Raspberry Pi.

\item[PQC] \textbf{Post-Quantum Cryptography.} Cryptographic algorithms designed to be secure against both classical and quantum computers.

\item[PSK] \textbf{Pre-Shared Key.} A secret known to both drone and GCS before any communication begins. Used in this system for HMAC-based drone authentication during the handshake.

\item[Rekey] The process of performing a new handshake to establish fresh encryption keys, replacing the current session keys. Prevents long-lived key compromise and provides forward secrecy.

\item[Replay Attack] An attack where a previously captured legitimate packet is retransmitted. Prevented in this system by a sliding window that tracks which sequence numbers have been seen.

\item[Selector] A system call mechanism (Python \texttt{selectors} module) for monitoring multiple I/O channels simultaneously. The proxy engine uses selectors instead of \texttt{asyncio} for deterministic latency behavior.

\item[Session ID] An 8-byte random value generated during each handshake. Included in every packet header to ensure packets from old sessions are rejected.

\item[SPHINCS+] A stateless hash-based digital signature scheme. Conservative post-quantum security based solely on hash function properties. Larger signatures than lattice-based schemes but fewer assumptions.

\item[Suite] In this system, a specific combination of (KEM, Signature, AEAD) algorithms. Example: \suiteid{cs-mlkem768-aesgcm-mldsa65} uses ML-KEM-768 for key exchange, AES-256-GCM for data encryption, and ML-DSA-65 for authentication.

\item[UAV] \textbf{Unmanned Aerial Vehicle.} A drone. In this system, a quadcopter equipped with a Pixhawk flight controller and a Raspberry Pi companion computer.

\item[UDP] \textbf{User Datagram Protocol.} A connectionless transport protocol. MAVLink and the encrypted data plane in this system use UDP because it provides low latency without the overhead of TCP's reliability mechanisms.

\item[Wire Format] The exact byte layout of data as it appears ``on the wire'' (transmitted over the network). This system's wire format is: Header (22 bytes) $\|$ Ciphertext $\|$ Authentication Tag.

% ── Concept-Level Entries ──────────────────────────────────────────────────

\item[Blackout Period] The interval during a rekey operation when no encrypted data is transmitted. While the old proxy is shutting down and the new proxy is performing its handshake, MAVLink packets are queued or dropped. Minimising this interval is a key engineering goal.

\item[Bump-in-the-Wire] A network security architecture where a transparent proxy is inserted into the communication path without modifying the endpoints. In this system, the PQC tunnel sits between MAVProxy and the network interface, encrypting traffic without changes to the flight controller firmware or GCS application.

\item[Controller--Follower] The benchmark scheduler's architecture pattern where one side (the drone, or ``controller'') makes all scheduling decisions and the other side (the GCS, or ``follower'') executes commands. This eliminates distributed consensus and simplifies error handling.

\item[CRQC] \textbf{Cryptographically Relevant Quantum Computer.} A quantum computer large enough to run Shor's algorithm against production-sized cryptographic keys (e.g.\ factoring 2048-bit RSA). No CRQC exists as of 2025, but the ``harvest now, decrypt later'' threat motivates PQC deployment today.

\item[Dolev--Yao Model] A formal adversary model for cryptographic protocol analysis where the attacker controls the entire communication channel: observing, injecting, modifying, replaying, delaying, and dropping messages. The attacker cannot break the cryptographic primitives themselves. This system's threat model (Section~\ref{sec:eng-threat-model}) is based on Dolev--Yao.

\item[Domain Separation] A cryptographic design principle where different uses of the same primitive are given distinct context strings, preventing cross-protocol attacks. In this system, HKDF key derivation includes the session ID, KEM name, and SIG name in the \texttt{info} parameter, ensuring that keys derived in different sessions or with different algorithms are cryptographically independent.

\item[Forward Secrecy] A property of a key agreement protocol where compromise of long-term keys does not reveal past session keys. In this system, each handshake uses ephemeral KEM keypairs; compromising the GCS signing key does not help decrypt previously recorded sessions.

\item[Harvest Now, Decrypt Later] An attack strategy where an adversary records encrypted traffic today and stores it until a quantum computer capable of breaking the encryption becomes available. Abbreviated HNDL. This is the primary threat motivating the use of post-quantum cryptography in this system.

\item[Hysteresis] A control-theory concept applied to the scheduling policy. The policy requires a telemetry condition to persist for a minimum duration (5 seconds for upgrades, 30 seconds for downgrades) before triggering a suite change, preventing rapid oscillation between suites due to transient sensor fluctuations.

\item[ManagedProcess] A cross-platform subprocess wrapper (\filename{core/process.py}) that ensures child processes are terminated when the parent exits. Uses Win32 Job Objects on Windows and \texttt{PR\_SET\_PDEATHSIG} on Linux.

\item[Operation Chronos] The system's clock synchronisation protocol: a 3-message NTP-lite handshake over the TCP control channel that estimates the clock offset between drone and GCS. Named for the Greek god of time. Used to correct one-way latency measurements.

\item[Trust Boundary] The logical perimeter between trusted and untrusted components. In this system, the trust boundary is the network interface: plaintext exists only on localhost (trusted), while encrypted traffic crosses the WiFi LAN (untrusted). See Section~\ref{sec:eng-trust-boundary}.

\item[Two-Phase Commit] A design pattern used by the \texttt{BenchmarkPolicy} class. Phase~1 (\texttt{evaluate()}) makes a decision (advance to the next suite or stay) without modifying state. Phase~2 (\texttt{confirm\_advance()}) commits the state change only after metrics have been safely recorded. This prevents metric mis-attribution.

\item[Validation Verdict] A pass/fail classification (Category~R in the metrics schema) applied to each suite record. A suite passes if the handshake succeeded, data-plane counters are non-zero, and the duration is within bounds. See Section~\ref{sec:metrics-failures} for how different failure classes affect the verdict.

\end{description}
