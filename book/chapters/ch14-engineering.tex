\chapter{Engineering Trade-Offs and Security Analysis}
\label{ch:engineering}

Building a system is about making choices under constraints.  Every design
decision in the PQC Secure MAVLink Tunnel involves a trade-off---between
security and performance, between generality and simplicity, between
correctness and speed.  This chapter examines these trade-offs honestly,
discusses the system's limitations, and analyses its security properties.

% ============================================================
\section{Performance Trade-Offs}
\label{sec:eng-perf}

\subsection{NIST Level vs.\ Latency}

The most fundamental trade-off is between security level and handshake
latency.  Higher NIST levels use larger keys and more complex operations:

\begin{table}[htbp]
\centering
\caption{Representative handshake times by NIST level (Raspberry Pi~5)}
\label{tab:eng-level-latency}
\begin{tabular}{@{}l c c c@{}}
\toprule
\textbf{NIST Level} & \textbf{ML-KEM + ML-DSA} & \textbf{HQC + Falcon} &
\textbf{McEliece + SPHINCS+} \\
\midrule
L1 & $\sim 15$\,ms   & $\sim 50$\,ms    & $\sim 2{,}000$\,ms \\
L3 & $\sim 25$\,ms   & $\sim 100$\,ms   & $\sim 15{,}000$\,ms \\
L5 & $\sim 40$\,ms   & $\sim 200$\,ms   & $\sim 45{,}000$\,ms \\
\bottomrule
\end{tabular}
\end{table}

\begin{keyinsight}
For a drone that rekeys every 110~seconds, even a 45-second handshake means
the tunnel is ``dark'' for 41\,\% of the cycle.  This is why the suite
registry provides 72~options: researchers can find the sweet spot for their
specific mission.
\end{keyinsight}

\subsection{KEM Family Characteristics}

\begin{description}
  \item[ML-KEM (CRYSTALS-Kyber)] Fastest overall.  Small keys
        ($\sim 800$--$1{,}568$~bytes public key).  Lattice-based.
        Best choice for resource-constrained devices.
  \item[HQC] Code-based.  Slower than ML-KEM but offers algorithm diversity
        (different mathematical hardness assumption).  Moderate key sizes
        ($\sim 2{,}000$--$8{,}000$~bytes).
  \item[Classic McEliece] Extremely large public keys
        ($\sim 260{,}000$--$1{,}300{,}000$~bytes at L1--L5).
        Very fast decapsulation but key generation and encapsulation are slow.
        The large keys dominate handshake time due to network transfer.
\end{description}

\subsection{AEAD Algorithm Comparison}

\begin{table}[htbp]
\centering
\caption{AEAD algorithm trade-offs}
\label{tab:eng-aead}
\begin{tabular}{@{}l c c l@{}}
\toprule
\textbf{Algorithm} & \textbf{Hardware Accel} & \textbf{Tag Size} & \textbf{Notes} \\
\midrule
AES-256-GCM         & AES-NI (x86), ARMv8-CE & 16\,B & Fastest with hardware support \\
ChaCha20-Poly1305   & None needed             & 16\,B & Constant-time; good on ARM without CE \\
ASCON-128a          & None                    & 16\,B & Lightweight; designed for constrained devices \\
\bottomrule
\end{tabular}
\end{table}

The Raspberry Pi~5 (Cortex-A76) has ARMv8 Crypto Extensions, making AES-GCM
competitive.  On older ARM devices without these extensions, ChaCha20 would
be faster.  ASCON is the NIST Lightweight standard and is designed for very
small processors (8-bit, 32-bit microcontrollers), so it is not the fastest
on a 64-bit ARM but provides algorithm diversity.

\subsection{Power vs.\ Security}

Post-quantum cryptographic operations consume measurably more power than their
classical equivalents.  The INA219 power monitor (Section~\ref{sec:metrics-ina219})
captures this precisely.  Key observations:

\begin{itemize}
  \item Power consumption during handshake is dominated by CPU-intensive
        KEM and SIG operations.
  \item ML-KEM handshakes on the Pi~5 consume $\sim 5$--$8\,$W peak (vs.\
        $\sim 3\,$W idle).
  \item Classic McEliece key generation can sustain peak power for tens of
        seconds, significantly impacting battery-powered drones.
  \item The \texttt{energy\_per\_handshake\_j} metric directly quantifies
        the cost of security.
\end{itemize}

% ============================================================
\section{Design Decisions and Their Rationale}
\label{sec:eng-decisions}

\subsection{Bump-in-the-Wire Architecture}

\begin{designdecision}
The proxy operates as a transparent bump-in-the-wire: it does not modify
MAVLink payloads, does not parse MAVLink semantics, and does not require
changes to the flight controller firmware or the GCS application.  This
maximises compatibility but means the proxy cannot perform MAVLink-aware
optimisations (e.g.\ prioritising heartbeat messages).
\end{designdecision}

\subsection{UDP over TCP}

\begin{designdecision}
The encrypted tunnel uses UDP, not TCP.  MAVLink is inherently a
best-effort protocol---it expects occasional packet loss and compensates with
redundant streams.  TCP's retransmission and head-of-line blocking would add
latency spikes that are worse for real-time control than losing a single
packet.  The AEAD layer provides integrity guarantees that TCP's checksums
would otherwise provide.
\end{designdecision}

\subsection{Selectors over asyncio}

\begin{designdecision}
The proxy uses \texttt{selectors.DefaultSelector} (which maps to
\texttt{epoll} on Linux) rather than Python's \texttt{asyncio} framework.
Selectors provide microsecond-level event notification with minimal overhead.
\texttt{asyncio} would add coroutine scheduling overhead and make the
event loop harder to reason about in a security-critical context.
\end{designdecision}

\subsection{No Key Caching}

\begin{designdecision}
Every suite switch performs a fresh KEM key generation, encapsulation, and
signature.  There is no session resumption or key caching.  This is a
deliberate choice for the benchmark: every measurement captures the full
cost of the PQC handshake.  A production system might cache session keys
for faster resumption.
\end{designdecision}

\subsection{Two-Phase Commit for Policy}

\begin{designdecision}
The \texttt{evaluate()} / \texttt{confirm\_advance()} split
(Section~\ref{sec:sched-2phase}) adds complexity but prevents a dangerous
class of bugs: metrics being attributed to the wrong suite when the index
advances before finalisation completes.
\end{designdecision}

% ============================================================
\section{Security Analysis}
\label{sec:eng-security}

\subsection{Formal Threat Model}
\label{sec:eng-threat-model}

This section specifies the adversary model, trust assumptions,
security goals, and explicit non-goals that bound the system's
security claims.

\subsubsection{Adversary Capabilities}

The system is designed to resist the following adversary classes:

\begin{description}
  \item[Passive network adversary (Eavesdropper).]
    Can observe all packets on the wireless LAN between drone and GCS.
    Can record ciphertext, headers, timing, and packet sizes.
    Cannot modify or inject traffic.

  \item[Active network adversary (Dolev--Yao style).]
    In addition to passive capabilities, can inject, modify, replay,
    delay, reorder, and selectively drop packets on the link between
    drone and GCS.  This is the standard Dolev--Yao network attacker
    model~\cite{dolev1983security}: the adversary controls the
    communication channel but not the endpoints.

  \item[Future quantum adversary (``harvest now, decrypt later'').]
    Records ciphertext today and stores it until a cryptographically
    relevant quantum computer (CRQC) exists.  At that point, the
    adversary applies Shor's algorithm~\cite{shor1997} to any
    classical public-key material and Grover's algorithm~\cite{grover1996}
    to symmetric keys.  The system's use of post-quantum KEMs and
    signatures is specifically motivated by this adversary.
\end{description}

\subsubsection{Trust Assumptions}

The security analysis depends on the following assumptions.  If any
assumption is violated, the corresponding security property may not hold.

\begin{enumerate}
  \item \textbf{Pre-deployed identity keys are authentic.}
        Signing key pairs (GCS) and pre-shared keys (drone PSK) are
        generated offline, transported physically or via a
        pre-authenticated channel, and stored on the local filesystem.
        The system does not establish key authenticity---it assumes it.

  \item \textbf{Endpoint hardware is not compromised.}
        The Raspberry Pi (drone) and the GCS laptop are assumed to be
        free of hardware implants, firmware backdoors, and rootkits.
        Physical tamper protection is out of scope.

  \item \textbf{Localhost is a trust boundary.}
        Plaintext MAVLink traffic travels exclusively over the loopback
        interface (\texttt{127.0.0.1}).  No unprivileged remote process
        can read localhost traffic.  If an attacker gains local code
        execution on either endpoint, the plaintext data plane is exposed.

  \item \textbf{The operating system provides process isolation.}
        Memory isolation between the proxy process and other processes
        is assumed to be intact (no shared-memory side channels,
        no \texttt{/proc/\{pid\}/mem} access by unprivileged users).

  \item \textbf{Cryptographic primitives are correct.}
        The implementations of ML-KEM, ML-DSA, Falcon, SPHINCS+, HQC,
        Classic McEliece (via \texttt{liboqs}), AES-GCM, ChaCha20-Poly1305
        (via the Python \texttt{cryptography} library), and ASCON-128a
        are assumed to be functionally correct and free of catastrophic
        implementation bugs.  No formal verification has been performed
        on these libraries.

  \item \textbf{Monotonic clocks are monotonic.}
        \texttt{time.monotonic()} and \texttt{time.perf\_counter\_ns()} are
        assumed to never go backwards.  Policy timing (hysteresis, intervals)
        depends on this property.
\end{enumerate}

\subsubsection{Security Goals}

Given the above adversary model and trust assumptions, the system aims
to provide:

\begin{enumerate}
  \item \textbf{Data-plane confidentiality}: An eavesdropper learns nothing
        about the plaintext content of MAVLink datagrams (beyond their
        encrypted length and timing).
  \item \textbf{Data-plane integrity}: Any modification to a ciphertext
        packet (header or payload) is detected and the packet is dropped.
  \item \textbf{GCS authentication}: The drone can verify that it is
        communicating with the holder of the GCS signing key.
  \item \textbf{Drone identity binding}: The GCS can verify that the drone
        possesses the pre-shared key (HMAC-PSK).
  \item \textbf{Replay protection}: Retransmission of previously captured
        valid packets is detected and rejected.
  \item \textbf{Session isolation}: Packets from a previous handshake session
        are rejected (via session ID and epoch checks).
  \item \textbf{Forward secrecy}: Compromise of long-term signing keys does
        not reveal past session keys (KEM keypairs are ephemeral).
  \item \textbf{Post-quantum resistance}: All of the above properties hold
        against a future quantum adversary.
\end{enumerate}

\subsubsection{Explicit Non-Goals}
\label{sec:eng-nongoals}

The following are \textbf{not} security goals of this system.  They are
listed explicitly to prevent misinterpretation of the system's claims.

\begin{enumerate}
  \item \textbf{Denial-of-service resilience.}
        The system does not guarantee availability.  An active network
        attacker can drop all packets, severing the link.  Rate limiting
        (Section~\ref{sec:proxy-rl}) provides limited protection against
        handshake flooding, but the data plane has no anti-DoS mechanism
        beyond the inherent tolerance of UDP-based MAVLink.

  \item \textbf{Traffic analysis resistance.}
        Packet lengths, timing, and rates are observable by a passive
        adversary.  The system does not pad packets, add dummy traffic,
        or otherwise obscure traffic patterns.  An adversary can infer
        activity levels and message types from packet sizes.

  \item \textbf{Insider threat protection.}
        An attacker with local code execution on either endpoint can
        read plaintext MAVLink traffic from the loopback interface, extract
        key material from process memory, or manipulate the proxy.

  \item \textbf{Physical tamper resistance.}
        No hardware security modules (HSMs), trusted platform modules (TPMs),
        or tamper-evident enclosures are used.  Keys reside as files on disk.

  \item \textbf{Key revocation and PKI.}
        There is no certificate authority, no certificate revocation
        mechanism, and no key expiry enforcement.  Revoking a compromised
        key requires manual re-provisioning.

  \item \textbf{Multi-party or multi-drone operation.}
        The system secures a single point-to-point link between one drone
        and one GCS.  Mesh networks, relay topologies, and multi-vehicle
        swarms are out of scope.

  \item \textbf{Constant-time guarantees.}
        The Python runtime does not provide constant-time execution.
        Timing side channels in KEM/SIG operations (via \texttt{liboqs})
        or in branch-dependent Python code are not mitigated.
\end{enumerate}

\subsubsection{Scope of the Data-Plane Security Boundary}
\label{sec:eng-trust-boundary}

Figure~\ref{fig:trust-boundary} illustrates the trust boundaries.

\begin{figure}[htbp]
\centering
\begin{tikzpicture}[
    zone/.style={draw, dashed, rounded corners=6pt, inner sep=10pt,
                 fill=#1, fill opacity=0.08, text opacity=1},
    box/.style={draw, rounded corners, minimum width=2.5cm,
                minimum height=0.7cm, font=\small, align=center},
    arr/.style={->, >=stealth, thick},
  ]
  % Trusted zone (left)
  \begin{scope}
    \node[box, fill=green!20] (fc)   at (0, 0)   {Flight Controller};
    \node[box, fill=green!20] (mav)  at (0,-1.5)  {MAVProxy};
    \node[box, fill=blue!20]  (pxd)  at (0,-3)    {Drone Proxy};
    \node[zone=green, fit=(fc)(mav)(pxd),
          label={[font=\footnotesize]above:Drone (localhost)}] {};
  \end{scope}

  % Untrusted zone (center)
  \node[box, fill=red!15, minimum width=3cm] (net) at (5.5,-1.5)
        {WiFi LAN\\(untrusted)};

  % Trusted zone (right)
  \begin{scope}
    \node[box, fill=blue!20]  (pxg)  at (11,-1.5)  {GCS Proxy};
    \node[box, fill=green!20] (gcs)  at (11, 0)    {Mission Planner};
    \node[box, fill=green!20] (sched) at (11,-3)   {Scheduler};
    \node[zone=green, fit=(pxg)(gcs)(sched),
          label={[font=\footnotesize]above:GCS (localhost)}] {};
  \end{scope}

  % Arrows
  \draw[arr, green!60!black] (fc)  -- node[right,font=\scriptsize]{serial} (mav);
  \draw[arr, green!60!black] (mav) -- node[right,font=\scriptsize]{UDP 127.0.0.1} (pxd);
  \draw[arr, red!60!black, line width=1.5pt]
        (pxd) -- node[above,font=\scriptsize]{encrypted UDP} (net);
  \draw[arr, red!60!black, line width=1.5pt]
        (net) -- node[above,font=\scriptsize]{encrypted UDP} (pxg);
  \draw[arr, green!60!black] (pxg) -- node[right,font=\scriptsize]{UDP 127.0.0.1} (gcs);
\end{tikzpicture}
\caption{Trust boundary diagram.  Green regions are trusted (localhost).
The red region (WiFi LAN) is the untrusted channel secured by the PQC
tunnel.  The bold red arrows represent AEAD-encrypted traffic.}
\label{fig:trust-boundary}
\end{figure}

The trust boundary has five key properties:

\begin{enumerate}
  \item \textbf{Plaintext isolation}: All plaintext traffic is bound to
        \texttt{127.0.0.1}.  No plaintext datagram ever crosses a network
        interface.

  \item \textbf{Encrypted-plane authentication}: Every packet crossing the
        untrusted network carries a 16-byte AEAD authentication tag over
        both the ciphertext and the 22-byte header (as AAD).

  \item \textbf{Session binding}: The 8-byte session ID in every packet
        header cryptographically binds data-plane traffic to a specific
        handshake instance.

  \item \textbf{Rate-limited handshake}: A token-bucket rate limiter
        (5~tokens, 1~refill/second) on the GCS handshake server prevents
        rapid reconnection attempts.

  \item \textbf{Strict peer matching}: When enabled
        (\configkey{STRICT\_UDP\_PEER\_MATCH}), the proxy rejects encrypted
        packets from any source IP/port other than the established peer.
\end{enumerate}

\begin{securitynote}
The \textbf{TCP control channel} (port~48080) used by the benchmark
scheduler is \textbf{outside} the security boundary.  It carries
plaintext JSON commands (start proxy, stop suite, clock sync) and is
neither encrypted nor authenticated.  This is acceptable for a lab
benchmark on a controlled LAN but would require TLS or HMAC
authentication in any deployment scenario.  Similarly, Operation Chronos
(clock synchronisation) runs over this same unauthenticated channel;
a network attacker could manipulate clock offsets to influence
time-based scheduling policy decisions.
\end{securitynote}

\subsection{Cryptographic Security Properties}

\begin{description}
  \item[Confidentiality] Provided by AEAD encryption (AES-256-GCM,
        ChaCha20-Poly1305, or ASCON-128a) with keys derived from a PQC KEM
        shared secret via HKDF-SHA256.

  \item[Integrity] AEAD tags (16~bytes) detect any modification to the
        ciphertext or header (which is used as AAD).

  \item[Authentication] The GCS signs its ServerHello with a PQC digital
        signature (ML-DSA, Falcon, or SPHINCS+).  The drone verifies the
        signature using the pre-deployed public key.  The drone authenticates
        to the GCS via HMAC-SHA256 over the shared secret, binding the PSK
        identity to the handshake.

  \item[Replay protection] An 8-byte sequence number and a 64-entry sliding
        bitmap window (Section~\ref{sec:aead-replay}) prevent replay attacks.
        Epoch numbers prevent cross-session replay.

  \item[Forward secrecy] Each handshake generates fresh KEM keypairs.
        Compromise of the long-term signing key does not reveal past session
        keys (assuming the KEM shared secret was properly destroyed).

  \item[Post-quantum security] The KEM algorithms (ML-KEM, HQC, Classic
        McEliece) are designed to resist quantum attacks.  The signature
        algorithms (ML-DSA, Falcon, SPHINCS+) are post-quantum secure.
        An attacker recording today's handshake cannot extract the shared
        secret even with a future quantum computer.
\end{description}

\subsection{Known Limitations and Residual Risks}

\begin{securitynote}
The following are \textbf{known limitations}, not bugs.  They represent
conscious design choices for a research benchmark system.

\begin{enumerate}
  \item \textbf{No mutual authentication in KEM}: Only the GCS is
        authenticated via digital signature.  The drone authenticates via
        HMAC-SHA256 over the PSK, which provides weaker identity binding than
        a full mutual signature exchange.

  \item \textbf{Plaintext control channel}: The TCP control channel
        (port~48080) is unencrypted.  An attacker could inject fake commands
        (e.g.\ force a suite switch, cause a shutdown).  Acceptable in a lab
        LAN; would need TLS or HMAC in production.

  \item \textbf{Clock sync is unauthenticated}: An attacker could manipulate
        Chronos timestamps to skew the clock offset, affecting time-based
        policy decisions.

  \item \textbf{No certificate infrastructure}: Signing keys are bare files
        on disk, not wrapped in certificates with expiry dates or revocation
        lists.

  \item \textbf{No Perfect Forward Secrecy for the PSK}: The drone PSK
        (pre-shared key) is static.  If compromised, an attacker could
        impersonate the drone in future handshakes (though not decrypt past
        sessions, since KEM keypairs are ephemeral).

  \item \textbf{Sequence number overflow}: The 8-byte sequence number
        supports $2^{64}$ packets.  At 1000~packets/second, overflow would
        take $\sim 585$~million years.  The system raises an exception
        before overflow as a defence-in-depth measure.

  \item \textbf{Side-channel attacks}: The PQC implementations from
        \texttt{liboqs} are not guaranteed to be constant-time on all
        platforms.  A co-located attacker with cache-timing access could
        potentially extract key material.
\end{enumerate}
\end{securitynote}

% ============================================================
\section{Cross-Platform Considerations}
\label{sec:eng-xplat}

The system runs on two very different platforms:

\begin{table}[htbp]
\centering
\caption{Platform comparison}
\label{tab:eng-platforms}
\begin{tabular}{@{}l l l@{}}
\toprule
\textbf{Aspect} & \textbf{Drone (Pi~5)} & \textbf{GCS (Windows)} \\
\midrule
OS         & Raspberry Pi OS (Debian) & Windows 10/11 \\
CPU        & ARM Cortex-A76, 4 cores  & x86-64, 8+ cores \\
RAM        & 4--8\,GB                 & 16--32\,GB \\
Python     & 3.11+                    & 3.11+ \\
Power mon  & INA219 / RPi5 hwmon      & Not available \\
Process mgmt & PDEATHSIG via libc     & Win32 Job Objects \\
MAVProxy   & Serial to flight controller & UDP from tunnel \\
\bottomrule
\end{tabular}
\end{table}

Key cross-platform challenges:

\begin{itemize}
  \item \textbf{liboqs availability}: The OQS library must be compiled
        natively on each platform.  ARM and x86 may support different
        algorithm subsets.
  \item \textbf{Path separators}: All paths use \texttt{pathlib.Path} to
        handle \texttt{/} vs.\ \texttt{\textbackslash} transparently.
  \item \textbf{Signal handling}: \texttt{SIGTERM} is not available on
        Windows in the same way.  \texttt{ManagedProcess} uses
        \texttt{TerminateProcess} as the Windows equivalent.
  \item \textbf{Console allocation}: MAVProxy's \texttt{prompt\_toolkit}
        requires a real Windows console.  The GCS server launches MAVProxy
        with \texttt{new\_console=True} when GUI is enabled.
\end{itemize}

% ============================================================
\section{Scalability and Resource Usage}
\label{sec:eng-scale}

\subsection{Memory}

The proxy's memory footprint is dominated by:
\begin{itemize}
  \item Socket buffers (4~UDP sockets, kernel-managed).
  \item AEAD key material (two 32-byte keys + nonce state).
  \item The replay window bitmap (64~bits = 8~bytes).
  \item Configuration dictionary ($\sim 100$~keys, $\sim 10$\,KB).
\end{itemize}

Total proxy memory is typically under 50\,MB.

\subsection{CPU}

During steady-state encrypted forwarding, the main costs are:
\begin{itemize}
  \item AEAD encryption (one per outbound packet).
  \item AEAD decryption (one per inbound packet).
  \item \texttt{select()} system call overhead.
\end{itemize}

At 20~MAVLink packets/second (typical for HEARTBEAT + telemetry), CPU usage
is negligible ($< 5$\,\%).  During handshake, CPU can spike to 100\,\% on one
core for KEM/SIG operations.

\subsection{Network Bandwidth}

The AEAD framing adds 22~bytes of header + 16~bytes of tag = 38~bytes
overhead per packet.  For a typical 263-byte MAVLink v2 packet, this is
$38 / 263 \approx 14.4\,$\% overhead.

% ============================================================
\section{Testing Strategy}
\label{sec:eng-testing}

The codebase includes multiple test layers:

\begin{enumerate}
  \item \textbf{Unit tests} (\texttt{tests/}): Test individual functions
        (suite registry, AEAD operations, config parsing).
  \item \textbf{Integration tests} (\texttt{test\_*.py} at repo root):
        Test the complete proxy loop (start proxy, perform handshake, send
        data, verify receipt).
  \item \textbf{Benchmark tests} (\texttt{test\_comprehensive\_benchmark.py}):
        Test the full benchmark scheduler with multiple suites.
  \item \textbf{Validation scripts} (\texttt{verify\_*.py}): Verify
        collector output, metrics consistency, and drone/GCS alignment.
  \item \textbf{The benchmark itself}: A 72-suite benchmark run is the
        ultimate integration test---if all suites pass, the entire pipeline
        (handshake, AEAD, proxy, scheduler, metrics, persistence) works
        end-to-end.
\end{enumerate}

% ============================================================
\section{Chapter Summary}

\begin{itemize}
  \item The fundamental trade-off is \textbf{security level vs.\ latency}:
        higher NIST levels and more conservative KEM families (McEliece)
        incur dramatically higher handshake times.
  \item Key design decisions (bump-in-wire, UDP, selectors, no key caching,
        two-phase commit) are justified by the system's requirements.
  \item The system provides \textbf{confidentiality, integrity,
        authentication, replay protection, forward secrecy, and post-quantum
        security}.
  \item Known limitations (unauthenticated control channel, no PKI,
        static PSK) are acceptable for a lab benchmark and are documented for
        future hardening.
  \item Cross-platform operation between Linux/ARM and Windows/x86 introduces
        challenges in process management, library availability, and console
        handling.
  \item Resource usage is modest: under 50\,MB RAM, $< 5$\,\% CPU at
        steady state, and $\sim 14$\,\% bandwidth overhead.
\end{itemize}
