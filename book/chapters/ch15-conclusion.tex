\chapter{Conclusion and Future Work}
\label{ch:conclusion}

% ============================================================
\section{Summary of Contributions}

This book has documented a complete, working Post-Quantum Cryptographic
(PQC) Secure MAVLink Tunnel system---from theoretical foundations through
implementation details to benchmark orchestration and visual analysis.
The key contributions are:

\begin{enumerate}
  \item \textbf{A bump-in-the-wire PQC tunnel}: A transparent proxy that
        encrypts MAVLink traffic between a drone and ground control station
        using post-quantum algorithms, without modifying the flight controller
        firmware or GCS application software.

  \item \textbf{72 cipher suites}: A comprehensive suite registry combining
        9~KEM algorithms, 8~signature algorithms, and 3~AEAD ciphers across
        NIST security levels~1, 3, and~5.  This provides researchers with
        the largest PQC suite matrix applied to drone communications to date.

  \item \textbf{A telemetry-aware scheduling policy}: A safety-critical
        state machine that adapts cryptographic strength to real-time
        conditions (battery, temperature, link quality) with hysteresis,
        blacklisting, and rate limiting.

  \item \textbf{An 18-category metrics pipeline}: A structured data
        collection framework with $\sim 160$~typed fields covering
        cryptographic primitives, data-plane counters, MAVLink integrity,
        power consumption, and system resources.

  \item \textbf{High-frequency power monitoring}: Direct INA219 register
        access at up to 1{,}100~samples/second for precise energy-per-handshake
        measurements.

  \item \textbf{A forensic analysis dashboard}: A web-based tool for
        multi-run comparison, anomaly detection, and visual exploration of
        benchmark results across 12~interactive pages.

  \item \textbf{Cross-platform operation}: The entire system runs on a
        Raspberry Pi~5 (drone) and a Windows laptop (GCS), demonstrating
        practical deployment on real constrained hardware.
\end{enumerate}

% ============================================================
\section{Key Findings}
\label{sec:concl-findings}

From the benchmark results (Chapter~\ref{ch:benchmarks}), several
findings emerge.  Table~\ref{tab:concl-summary} summarises the
headline numbers.

\begin{longtable}{p{4cm} r r r}
  \caption{Headline benchmark results (median, $n=200$, RPi~4 Cortex-A72).}
  \label{tab:concl-summary} \\
  \toprule
  \textbf{Metric} & \textbf{Best} & \textbf{Worst} & \textbf{Ratio} \\
  \midrule
  \endfirsthead
  \bottomrule
  \endfoot

  KEM keygen (ms)     & 0.082 (ML-KEM-512) & 7{,}066 (McE-8192128) & 86{,}000$\times$ \\
  SIG sign (ms)       & 0.641 (Falcon-512)  & 2{,}598 (SPHINCS$^+$-192s) & 4{,}050$\times$ \\
  AEAD encrypt ($\mu$s, 64\,B) & 4.15 (Ascon)  & 7.28 (AES-GCM)       & 1.8$\times$ \\
  Suite handshake (ms) & $\sim$2 (ML-KEM-768+ML-DSA-65) & 12{,}377 (McE-8M+SPHINCS$^+$-256s) & 6{,}000$\times$ \\
  KEM energy ($\mu$J) & 876 (ML-KEM-512 encaps) & 27.6\,M (McE-8192128 keygen) & 31{,}500$\times$ \\
  SIG energy ($\mu$J) & 741 (Falcon-512 verify) & 11.3\,M (SPHINCS$^+$-192s sign) & 15{,}200$\times$ \\
\end{longtable}

From these measurements:

\begin{enumerate}
  \item \textbf{ML-KEM dominates}: ML-KEM (CRYSTALS-Kyber) consistently
        delivers the lowest handshake latency across all NIST levels, making
        it the clear choice for latency-sensitive drone applications.

  \item \textbf{Classic McEliece is impractical for real-time rekey}:
        While offering strong security guarantees based on a different
        mathematical problem (coding theory), McEliece's enormous public keys
        (up to 1.3\,MB) make handshakes take tens of seconds, far too slow
        for operational drone rekey intervals.

  \item \textbf{AEAD choice matters less than KEM choice}: The three AEAD
        algorithms (AES-GCM, ChaCha20, ASCON) differ by microseconds per
        packet, while KEM choice affects handshake time by milliseconds to
        seconds.

  \item \textbf{Power cost is measurable but manageable}: PQC handshakes
        on the Pi~5 consume 5--8\,W peak, compared to 3\,W idle.  For a
        typical 110-second suite interval, the handshake energy is a small
        fraction of total energy.

  \item \textbf{Algorithm diversity has value}: HQC and SPHINCS+ provide
        security from fundamentally different mathematical assumptions
        (coding theory and hash functions, respectively), hedging against
        the possibility that lattice-based cryptography is broken.
\end{enumerate}

% ============================================================
\section{Lessons Learned}
\label{sec:concl-lessons}

\begin{enumerate}
  \item \textbf{Metrics must be crash-resilient}: Early versions lost
        all data when a suite crashed.  The RobustLogger's append-mode
        design (Section~\ref{sec:metrics-robust}) solved this.

  \item \textbf{Time domains must not mix}: Using wall-clock time where
        monotonic time is expected (or vice versa) produced subtly wrong
        interval calculations.  Consistent use of \texttt{time.monotonic()}
        for policy timing eliminated this class of bugs.

  \item \textbf{Two-phase commit prevents metric mis-attribution}: The
        evaluate/confirm\_advance split (Section~\ref{sec:sched-2phase})
        was introduced after discovering that metrics were being recorded
        against the wrong suite.

  \item \textbf{Cross-platform process management is harder than expected}:
        Orphaned proxy processes on both Linux and Windows required
        platform-specific solutions (PDEATHSIG, Win32 Job Objects).

  \item \textbf{PQC library maturity varies}: Some algorithms in
        \texttt{liboqs} have performance regressions between versions, and
        not all algorithms are available on all platforms.  Runtime probing
        (Section~\ref{sec:suites-probing}) is essential.
\end{enumerate}

% ============================================================
\section{Future Work}
\label{sec:concl-future}

\subsection{Short-Term Improvements}

\begin{enumerate}
  \item \textbf{TLS for the control channel}: Replace the plaintext TCP
        control channel with mutual TLS or HMAC-authenticated messages.

  \item \textbf{Certificate-based identity}: Replace bare signing key files
        with X.509 certificates (or a lightweight equivalent) that include
        expiry dates and support revocation.

  \item \textbf{Session resumption}: Cache KEM shared secrets (with a TTL)
        to enable faster rekey without a full handshake.

  \item \textbf{Hybrid classical/PQC}: Combine a classical ECDH exchange
        with a PQC KEM to provide security even if one is broken (``belt and
        suspenders'' approach, as recommended by NIST).
\end{enumerate}

\subsection{Medium-Term Research Directions}

\begin{enumerate}
  \item \textbf{Real flight testing}: Run the benchmark during actual drone
        flights to measure the impact of vibration, altitude, and RF
        interference on PQC tunnel performance.

  \item \textbf{Multi-hop tunnels}: Extend the architecture to support mesh
        networks where multiple drones relay encrypted traffic.

  \item \textbf{Machine-learning policy}: Replace the rule-based
        TelemetryAwarePolicyV2 with a reinforcement-learning agent that
        optimises suite selection based on historical performance data.

  \item \textbf{Hardware acceleration}: Evaluate FPGA or GPU acceleration
        for lattice-based operations (NTT, polynomial multiplication) on
        the drone side.
\end{enumerate}

\subsection{Long-Term Vision}

\begin{enumerate}
  \item \textbf{Standardisation}: Contribute the handshake protocol and
        wire format to MAVLink standardisation efforts as a PQC security
        extension.

  \item \textbf{Embedded implementation}: Port the critical path (handshake
        + AEAD) to C/Rust for deployment on microcontroller-based flight
        controllers (STM32, ESP32).

  \item \textbf{Formal verification}: Use tools like ProVerif or Tamarin to
        formally verify the handshake protocol's security properties.
\end{enumerate}

% -- Roadmap Figure --
\begin{figure}[htbp]
\centering
\begin{tikzpicture}[
  phase/.style={draw, rounded corners=3pt, minimum width=3.5cm, minimum height=0.8cm,
                font=\small, align=center, fill=#1},
  arr/.style={-{Stealth[length=2mm]}, thick, gray},
]
  \node[phase=green!20] (s1) at (0, 0) {Short-Term\\(3--6 months)};
  \node[phase=yellow!20] (s2) at (5, 0) {Medium-Term\\(6--18 months)};
  \node[phase=orange!20] (s3) at (10, 0) {Long-Term\\(1--3 years)};

  \draw[arr] (s1) -- (s2);
  \draw[arr] (s2) -- (s3);

  \node[font=\tiny, below=0.3cm of s1, text width=3.2cm, align=center] {
    TLS control channel\\
    Certificate identity\\
    Session resumption\\
    Hybrid PQC
  };
  \node[font=\tiny, below=0.3cm of s2, text width=3.2cm, align=center] {
    Real flight testing\\
    Multi-hop mesh\\
    ML-based policy\\
    FPGA acceleration
  };
  \node[font=\tiny, below=0.3cm of s3, text width=3.2cm, align=center] {
    MAVLink standard\\
    C/Rust embedded\\
    Formal verification
  };
\end{tikzpicture}
\caption{Future work roadmap.}
\label{fig:roadmap}
\end{figure}

% ============================================================
\section{Closing Remarks}

The transition to post-quantum cryptography is not a distant future
concern---it is an active engineering challenge that must be addressed
\emph{now}, before quantum computers become capable of breaking classical
encryption.  The ``harvest now, decrypt later'' threat means that drone
telemetry recorded today could be decrypted by a future adversary.

This system demonstrates that post-quantum protection for drone
communications is \emph{practical} on commodity hardware.  A Raspberry Pi~5
running Python can perform ML-KEM-512 handshakes in under 15~milliseconds,
encrypt MAVLink packets with negligible overhead, and cycle through 72~cipher
suites in a single automated benchmark run.

The code, the metrics, the analysis tools, and this book are all
contributions toward a future where unmanned systems communicate securely,
even in a post-quantum world.
