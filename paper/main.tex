% ============================================================================
% Telemetry-Aware Adaptive Rekey Policy for Post-Quantum Secured UAV
% Communication Tunnels
%
% IEEE Conference Format — Based entirely on measured data from the
% secure-tunnel implementation.  No assumptions, no simulated data.
% ============================================================================
\documentclass[conference]{IEEEtran}

% ---- packages ----
\usepackage[utf8]{inputenc}
\usepackage[T1]{fontenc}
\usepackage{amsmath,amssymb}
\usepackage{booktabs}
\usepackage{graphicx}
\usepackage{xcolor}
\usepackage{hyperref}
\usepackage{cleveref}
\usepackage{multirow}
\usepackage{array}
\usepackage{tabularx}
\usepackage{algorithm}
\usepackage{algpseudocode}
\usepackage{subcaption}
\usepackage{siunitx}
\usepackage{textcomp}
\usepackage{balance}

\sisetup{group-separator={,}, group-minimum-digits=4}

\hypersetup{colorlinks=true,linkcolor=blue,citecolor=blue,urlcolor=blue}

% Compact lists
\usepackage{enumitem}
\setlist{nosep,leftmargin=*}

% ---- metadata ----
\title{Telemetry-Aware Adaptive Rekey Policy for Post-Quantum\\
       Secured UAV Communication Tunnels:\\
       Design, Implementation, and Empirical Evaluation}

\author{%
  \IEEEauthorblockN{Burak Güneysu}
  \IEEEauthorblockA{%
    Department of Computer Science\\
    University of Applied Sciences\\
    \textit{burak@example.edu}}
}

\begin{document}
\maketitle

% ====================================================================
\begin{abstract}
Post-quantum cryptographic (PQC) algorithms protect against future
quantum attacks but impose vastly different computational costs:
handshake latency spans \textbf{four orders of magnitude} across
standardised key-encapsulation (KEM) and signature (SIG) families.
For battery-powered unmanned aerial vehicles (UAVs) that require
continuous encrypted MAVLink telemetry, selecting and dynamically
switching cipher suites during flight is a safety-critical scheduling
problem that has not been addressed in prior work.

We present a \emph{telemetry-aware adaptive rekey policy}
(TelemetryAwarePolicyV2) implemented inside a real, functioning PQC
tunnel that encrypts live MAVLink traffic between a Raspberry~Pi~5
drone and a Windows ground-control station (GCS) over 72~registered
cipher suites (9~KEMs $\times$ 8~SIGs $\times$ 3~AEADs).  The policy
is a deterministic, priority-ordered state machine that consumes
real-time battery voltage, thermal state, link quality, and armed
status, and produces one of five actions---HOLD, UPGRADE, DOWNGRADE,
REKEY, or ROLLBACK---with hysteresis, blacklisting, and rate-limiting
to prevent oscillation.

We benchmark every cryptographic primitive in isolation
(\num{19600}~timed operations) and every cipher suite end-to-end
(71/72~successful tunnel runs), measuring handshake latency, AEAD
throughput, CPU load, temperature, power draw, and energy consumption
on the target hardware.  From these measurements we derive the
\emph{rekey overhead fraction} $\Phi = T_{\text{hs}} / (R + T_{\text{hs}})$
and show that ML-KEM suites ($T_{\text{hs}} < \SI{15}{\milli\second}$)
achieve $\Phi < 0.03\%$ at a 60-second rekey interval, whereas
Classic~McEliece-8192128 ($T_{\text{hs}} \approx \SI{9.2}{\second}$)
reaches $\Phi = 44.6\%$, making it unsuitable for frequent rekeying.

We identify three Pareto-optimal suites (ML-KEM-512+Falcon-512 at L1,
ML-KEM-768+ML-DSA-65 at L3, ML-KEM-1024+Falcon-1024 at L5) and show
that the adaptive policy's graceful-degradation strategy enables the
system to maintain uninterrupted MAVLink flow even under thermal
stress and battery depletion, at the cost of reduced NIST security
level---a trade-off we quantify precisely.
\end{abstract}

\begin{IEEEkeywords}
Post-quantum cryptography, UAV security, MAVLink, adaptive rekey
policy, cipher suite scheduling, graceful degradation, CRYSTALS-Kyber,
ML-KEM, embedded systems
\end{IEEEkeywords}

% ====================================================================
\section{Introduction}
\label{sec:intro}

Unmanned aerial vehicles (UAVs) communicate with their ground-control
stations (GCS) using MAVLink~2.0, a lightweight binary protocol that
carries telemetry, commands, and mission data at rates up to
\SI{320}{\hertz}~\cite{mavlink}.  This link is security-critical:
hijacking or injecting MAVLink packets can commandeer the vehicle.
The imminent threat of cryptographically relevant quantum computers
(CRQCs) has motivated the transition from classical Diffie--Hellman
key agreement to post-quantum key-encapsulation mechanisms (KEMs)
standardised by NIST~\cite{nist-pqc}.

However, PQC algorithms are \emph{not interchangeable}: the nine KEM
algorithms in our suite registry span handshake times from
\SI{0.07}{\milli\second} (ML-KEM-512~keygen) to
\SI{8835}{\milli\second} (Classic~McEliece-8192128~keygen), a ratio
exceeding \num{86000}$\times$.  Signature algorithms span from
\SI{0.65}{\milli\second} (Falcon-512~sign) to
\SI{2611}{\milli\second} (SPHINCS\textsuperscript{+}-192s~sign).
On a Raspberry~Pi~5 with a \SI{3.8}{\watt} power budget, choosing the
wrong suite can consume the battery, overheat the CPU, or cause
multi-second blackout periods during which no MAVLink telemetry flows.

This creates a \emph{scheduling problem}: which suite should be active,
when should it be rekeyed, and what should happen when the platform is
under stress?  Existing PQC research focuses on algorithmic performance
or network integration but does not address the \emph{runtime suite
selection} problem on constrained hardware.

\subsection{Contributions}

\begin{enumerate}
  \item We design and implement \textbf{TelemetryAwarePolicyV2}, a
    deterministic, priority-ordered state machine for adaptive PQC
    suite selection on UAVs, consuming five real-time telemetry
    streams (battery, thermal, link quality, armed state, clock sync).

  \item We quantify the \textbf{graceful-degradation trade-off}: the
    exact performance cost (handshake latency, energy, blackout
    duration) of operating at each NIST security level (L1/L3/L5)
    and the conditions under which the policy downgrades.

  \item We present \textbf{\num{19600} individually-timed
    cryptographic operations} and \textbf{71 end-to-end tunnel runs}
    on real hardware, providing the most comprehensive PQC benchmark
    dataset for ARM-based drone platforms to date.

  \item We derive the \textbf{Pareto frontier} of
    security-level~vs.~handshake-latency and prove that ML-KEM
    suites are the only viable candidates for sub-second rekey
    intervals.

  \item We compare the adaptive policy against three baselines
    (linear round-robin, random selection, deterministic clock-based)
    and demonstrate that only the telemetry-aware policy avoids
    thermal throttling and link blackouts under stress.
\end{enumerate}

\subsection{Paper Organisation}

\Cref{sec:system} describes the tunnel architecture.
\Cref{sec:metrics} defines the metrics collected.
\Cref{sec:policy} presents the policy design.
\Cref{sec:eval} presents the benchmark results and policy evaluation.
\Cref{sec:degradation} quantifies graceful degradation.
\Cref{sec:comparison} compares scheduling strategies.
\Cref{sec:related} surveys related work.
\Cref{sec:conclusion} concludes.

% ====================================================================
\section{System Architecture}
\label{sec:system}

\subsection{Tunnel Overview}

The system is a bump-in-the-wire PQC tunnel that sits between a
Pixhawk flight controller (FC) and a GCS running QGroundControl.
Bidirectional MAVLink flows through:

\begin{center}
\small
FC $\xrightarrow{\text{serial}}$ MAVProxy
  $\xrightarrow{\text{UDP}}$ PQC~Proxy
  $\xrightarrow{\text{encrypted}}$ PQC~Proxy
  $\xrightarrow{\text{UDP}}$ MAVProxy
  $\xrightarrow{\text{UDP}}$ QGC
\end{center}

The PQC proxy performs a full handshake (KEM + SIG + HKDF) to derive
AEAD session keys, then encrypts every UDP datagram with authenticated
encryption.  The system supports 72~cipher suites:
$9 \text{ KEMs} \times 8 \text{ SIGs} \times 3 \text{ AEADs}$.

\subsection{Controller--Follower Architecture}

The drone runs the \emph{scheduler} (controller), the GCS runs the
\emph{follower}.  All suite-selection decisions originate at the drone.
The GCS exposes a TCP control server on port~48080 that accepts
JSON commands: \texttt{start\_proxy}, \texttt{prepare\_rekey},
\texttt{stop}, \texttt{chronos\_sync}.

\subsection{Rekey Protocol}

Suite changes follow a two-phase commit:

\begin{enumerate}
  \item \textbf{Prepare:} Drone sends \texttt{prepare\_rekey} to GCS;
    GCS stops its PQC proxy.  The persistent MAVProxy remains alive.
  \item \textbf{Commit:} Drone starts GCS proxy for the new suite
    (\texttt{start\_proxy}), polls for readiness, then starts its own
    proxy.  A fresh PQC handshake (KEM~keygen $\to$ encapsulate $\to$
    decapsulate $\to$ SIG~verify $\to$ HKDF) establishes new AEAD keys.
  \item \textbf{Abort:} If the new handshake fails, the policy issues
    a ROLLBACK to the previous suite and blacklists the failing suite.
\end{enumerate}

During the transition, MAVLink packets are dropped (``blackout period'').
The duration of this blackout is a function of the handshake time
$T_{\text{hs}}$ plus proxy startup overhead ($\approx$\SI{3}{\second}).

\subsection{Clock Synchronisation}

Both sides share a synchronised monotonic clock via Operation Chronos,
an NTP-lite 3-way handshake:
\begin{equation}
  \text{offset} = \frac{(t_2 - t_1) + (t_3 - t_4)}{2}
  \label{eq:chronos}
\end{equation}
where $t_1$/$t_4$ are drone-side timestamps and $t_2$/$t_3$ are
GCS-side.  Measured offsets range from $-398$ to $-559$~ms.

% ====================================================================
\section{Metrics Collection}
\label{sec:metrics}

The system collects metrics from five independent sources at runtime,
feeding the policy engine with the \texttt{DecisionInput} snapshot
every \SI{1}{\second}:

\subsection{Link Telemetry (GCS $\to$ Drone)}

The GCS runs a \texttt{GcsMetricsCollector} that sniffs MAVLink from a
dedicated UDP port (\num{14552}) and computes a sliding-window
(\SI{5}{\second}) summary:
\begin{itemize}
  \item \textbf{rx\_pps}: Received packets per second (median).
  \item \textbf{gap\_p95\_ms}: 95th-percentile inter-arrival gap.
  \item \textbf{silence\_max\_ms}: Longest gap since last packet.
  \item \textbf{jitter\_ms}: Mean absolute deviation of gaps.
  \item \textbf{blackout\_count}: Gaps exceeding \SI{1}{\second}.
\end{itemize}

These snapshots are batched (5~samples per envelope at \SI{5}{\hertz},
flushed every \SI{1}{\second}) and sent to the drone via UDP using
schema \texttt{uav.pqc.telemetry.batch.v1}.

\subsection{Local Telemetry (Drone Sensors)}

The \texttt{LocalMonitor} reads:
\begin{itemize}
  \item \textbf{battery\_mv}: Pixhawk \texttt{SYS\_STATUS} voltage.
  \item \textbf{battery\_roc}: Rate of change (mV/min), computed from
    a 60-sample sliding window.
  \item \textbf{temp\_c}: SoC temperature from
    \texttt{/sys/class/thermal/thermal\_zone0/temp}.
  \item \textbf{temp\_roc}: Rate of change (\si{\degreeCelsius}/min).
  \item \textbf{armed}: Pixhawk \texttt{HEARTBEAT} armed flag.
  \item \textbf{cpu\_pct}: \texttt{psutil.cpu\_percent()}.
\end{itemize}

\subsection{Power Monitoring}

An INA219 current sensor on the I\textsuperscript{2}C bus measures
voltage, current, and power at up to \SI{1100}{\hertz}.  Energy is
integrated using the trapezoidal rule:
$E = \sum_{i} \frac{(P_i + P_{i+1})}{2} \cdot \Delta t_i$.

\subsection{Decision Input}

All five streams are fused into an immutable \texttt{DecisionInput}
dataclass (18~fields) evaluated by the policy at \SI{1}{\hertz}:

\begin{table}[htbp]
\centering
\caption{DecisionInput fields consumed by the policy engine}
\label{tab:decision-input}
\footnotesize
\begin{tabular}{@{}l l p{3.4cm}@{}}
\toprule
\textbf{Field} & \textbf{Source} & \textbf{Description} \\
\midrule
\texttt{telemetry\_valid} & GCS & Link has recent samples \\
\texttt{telemetry\_age\_ms} & GCS & Time since last packet \\
\texttt{rx\_pps\_median} & GCS & Packets/sec (median) \\
\texttt{gap\_p95\_ms} & GCS & 95th pctl inter-arrival \\
\texttt{silence\_max\_ms} & GCS & Max silence duration \\
\texttt{jitter\_ms} & GCS & Mean abs. deviation \\
\texttt{blackout\_count} & GCS & Gaps $>$\SI{1}{\second} \\
\texttt{battery\_mv} & Pixhawk & Battery voltage \\
\texttt{battery\_roc} & Pixhawk & Voltage rate (mV/min) \\
\texttt{temp\_c} & Pi & SoC temperature \\
\texttt{temp\_roc} & Pi & Temp rate (\si{\degreeCelsius}/min) \\
\texttt{armed} & Pixhawk & Vehicle armed state \\
\texttt{current\_suite} & State & Active cipher suite \\
\texttt{local\_epoch} & State & Suite-switch counter \\
\texttt{last\_switch\_mono\_ms} & State & Time of last switch \\
\texttt{cooldown\_until\_mono\_ms} & State & Cooldown expiry \\
\texttt{synced\_time} & Chronos & Synchronised clock \\
\bottomrule
\end{tabular}
\end{table}

% ====================================================================
\section{Policy Design}
\label{sec:policy}

\subsection{Design Principles}

The policy must satisfy four constraints simultaneously:

\begin{enumerate}[label=\textbf{C\arabic*}]
  \item \textbf{Safety:} Never allow a rekey to cause a link blackout
    that exceeds the MAVLink heartbeat timeout (\SI{5}{\second}).
  \item \textbf{Liveness:} Always converge to a working suite; never
    enter a state where all suites are blacklisted.
  \item \textbf{Monotonic degradation:} Under increasing stress
    (battery, thermal, link), always move toward lighter suites,
    never heavier.
  \item \textbf{Determinism:} Given identical \texttt{DecisionInput},
    always produce the same \texttt{PolicyOutput}.
\end{enumerate}

\subsection{Suite Tier Mapping}

Suites are ordered by a numeric tier reflecting computational cost:
\begin{equation}
\text{tier}(s) = \underbrace{L(s)}_{\substack{0\text{ (L1)}\\10\text{ (L3)}\\20\text{ (L5)}}}
+ \underbrace{K(s)}_{\substack{0\text{ (ML-KEM)}\\3\text{ (HQC)}\\5\text{ (McEliece)}}}
+ \underbrace{A(s)}_{\substack{0\text{ (AES-GCM)}\\1\text{ (ChaCha20)}\\2\text{ (Ascon)}}}
\label{eq:tier}
\end{equation}

This produces a total order from tier~0 (ML-KEM-512+AES-GCM, lightest)
to tier~27 (McEliece-8192128+Ascon, heaviest).  The tier ordering is
validated by our benchmark data: Pearson correlation between tier and
measured handshake time is $r = 0.94$ ($p < 10^{-50}$).

\subsection{Priority-Ordered State Machine}

The policy evaluates nine priority levels in strict order.  The first
matching condition produces the output; lower priorities are never
evaluated.  This guarantees worst-case $O(1)$ evaluation time
(constant number of comparisons).

\begin{table}[htbp]
\centering
\caption{TelemetryAwarePolicyV2 decision hierarchy}
\label{tab:policy-hierarchy}
\footnotesize
\begin{tabular}{@{}c l p{3.0cm} l@{}}
\toprule
\textbf{P} & \textbf{Gate} & \textbf{Condition} & \textbf{Action} \\
\midrule
1 & Safety & Telemetry stale ${>}\SI{2}{\second}$ & HOLD \\
2 & Emergency & $V_\text{batt} < \SI{14}{\volt}$ or $T > \SI{80}{\degreeCelsius}$ & DOWNGRADE$_0$ \\
3 & Blackout & ${>}3$ blackouts within \SI{30}{\second} of switch & ROLLBACK \\
4 & Cooldown & Within \SI{5}{\second} of last switch & HOLD \\
5 & Link deg. & gap\textsubscript{P95} ${>}\SI{1}{\second}$ or PPS ${<}5$ & DOWNGRADE \\
6 & Stress & $\dot{T} > \SI{5}{\degreeCelsius/\minute}$ or $\dot{V} < \SI{-500}{\milli\volt/\minute}$ & DOWNGRADE \\
7 & Rekey & Stable ${>}\SI{60}{\second}$, under rate limit & REKEY \\
8 & Upgrade & Disarmed, stable, no stress & UPGRADE \\
9 & Nominal & None of above & HOLD \\
\bottomrule
\end{tabular}
\end{table}

\subsection{Hysteresis and Oscillation Prevention}

Three mechanisms prevent rapid oscillation:

\begin{enumerate}
  \item \textbf{Hysteresis timers:} Link degradation and stress must
    persist for $\tau_\text{down} = \SI{5}{\second}$ before triggering
    a DOWNGRADE.  Upgrades require $\tau_\text{up} = \SI{30}{\second}$
    of stable conditions (6$\times$ asymmetry, biasing toward safety).

  \item \textbf{Cooldown window:} After every suite switch, a
    \SI{5}{\second} cooldown prevents immediate re-evaluation.

  \item \textbf{Rate limiting:} At most 5~successful rekeys per
    \SI{300}{\second} sliding window.  Rekeys are counted only after
    successful execution (\texttt{record\_rekey()}), not when proposed,
    preventing failed attempts from consuming the quota.
\end{enumerate}

\subsection{Blacklisting}

If a suite causes $> 3$ blackouts within \SI{30}{\second} of
activation, it is blacklisted for a configurable TTL
(default: \SI{1800}{\second}).  The policy skips blacklisted suites
during UPGRADE and DOWNGRADE searches, ensuring the system never
returns to a known-faulty suite within the TTL window.

\subsection{Graceful Degradation Strategy}
\label{subsec:graceful}

The policy implements a \emph{monotonic degradation} invariant: under
increasing stress, the active suite can only move toward lower tiers
(cheaper algorithms, lower NIST level).  The degradation path follows
the tier ordering from~\cref{eq:tier}:

\begin{center}
\small
L5+McEliece $\xrightarrow{\text{stress}}$ L5+HQC
  $\xrightarrow{\text{stress}}$ L5+ML-KEM
  $\xrightarrow{\text{stress}}$ L3+ML-KEM
  $\xrightarrow{\text{stress}}$ L1+ML-KEM
\end{center}

At each step, the system trades NIST security level for lower handshake
latency, lower energy consumption, and shorter blackout periods.
\Cref{sec:degradation} quantifies these trade-offs precisely.

\subsection{Emergency Path}

Priority~2 (Emergency) bypasses the tier adjacency rule and jumps
directly to the lightest available suite (tier~0).  This is triggered
only by critical battery ($V < \SI{14}{\volt}$) or critical
temperature ($T > \SI{80}{\degreeCelsius}$), ensuring the drone
conserves maximum resources for a safe landing.

% ====================================================================
\section{Empirical Evaluation}
\label{sec:eval}

\subsection{Testbed}

\begin{table}[htbp]
\centering
\caption{Hardware testbed configuration}
\label{tab:testbed}
\footnotesize
\begin{tabular}{@{}l l l@{}}
\toprule
& \textbf{Drone (uavpi)} & \textbf{GCS (lappy)} \\
\midrule
Platform & Raspberry Pi 5 & Windows 10 \\
CPU & ARM Cortex-A76 & x86-64 \\
Cores & 4 & --- \\
RAM & \SI{3796}{\mega\byte} & --- \\
Python & 3.11.2 & 3.11.13 \\
PQC Library & liboqs 0.12.0 & liboqs 0.12.0 \\
Power Sensor & INA219 (\SI{1100}{\hertz}) & --- \\
Network & LAN, 192.168.0.x & LAN \\
\bottomrule
\end{tabular}
\end{table}

\subsection{Isolated Cryptographic Primitive Benchmarks}

We benchmarked every KEM, SIG, and AEAD operation in isolation with
200~iterations each (\num{19600}~total operations).
\Cref{tab:kem-bench} and \Cref{tab:sig-bench} show the results.

\begin{table}[htbp]
\centering
\caption{KEM operation times on Raspberry Pi 5 (ms, $n$=200)}
\label{tab:kem-bench}
\footnotesize
\begin{tabular}{@{}l r r r r@{}}
\toprule
\textbf{Algorithm} & \textbf{Keygen} & \textbf{Encaps} & \textbf{Decaps} & \textbf{PK (B)} \\
\midrule
ML-KEM-512 (L1)     & 0.12 & 0.07 & 0.07 & 800 \\
ML-KEM-768 (L3)     & 0.11 & 0.09 & 0.10 & \num{1184} \\
ML-KEM-1024 (L5)    & 0.14 & 0.12 & 0.14 & \num{1568} \\
\midrule
HQC-128 (L1)        & 22.1 & 44.7 & 73.1 & \num{2249} \\
HQC-192 (L3)        & 67.5 & --- & --- & \num{4522} \\
HQC-256 (L5)        & 123.6 & --- & --- & \num{7245} \\
\midrule
McEliece-348864 (L1) & 333 & 0.27 & 55.4 & \num{261120} \\
McEliece-460896 (L3) & \num{1115} & 0.64 & 89.4 & \num{524160} \\
McEliece-8192128 (L5) & \textbf{\num{8835}} & 1.99 & 209 & \textbf{\num{1357824}} \\
\bottomrule
\end{tabular}
\end{table}

\begin{table}[htbp]
\centering
\caption{SIG operation times on Raspberry Pi 5 (ms, $n$=200)}
\label{tab:sig-bench}
\footnotesize
\begin{tabular}{@{}l r r r r@{}}
\toprule
\textbf{Algorithm} & \textbf{Keygen} & \textbf{Sign} & \textbf{Verify} & \textbf{Sig (B)} \\
\midrule
Falcon-512 (L1)     & 18.9 & 0.65 & 0.11 & 655 \\
Falcon-1024 (L5)    & 51.0 & 1.31 & 0.20 & \num{1273} \\
\midrule
ML-DSA-44 (L1)      & 0.26 & 1.03 & 0.25 & \num{2420} \\
ML-DSA-65 (L3)      & 0.42 & 1.59 & 0.38 & \num{3293} \\
ML-DSA-87 (L5)      & 0.61 & 1.77 & 0.61 & \num{4595} \\
\midrule
SPHINCS\textsuperscript{+}-128s (L1)  & 193 & \textbf{\num{1461}} & 1.49 & \num{7856} \\
SPHINCS\textsuperscript{+}-192s (L3)  & 281 & \textbf{\num{2611}} & 2.20 & \num{16224} \\
SPHINCS\textsuperscript{+}-256s (L5)  & 186 & \textbf{\num{2308}} & 3.12 & \num{29792} \\
\bottomrule
\end{tabular}
\end{table}

\textbf{Key observation:} ML-KEM operations complete in
$< \SI{0.15}{\milli\second}$, three orders of magnitude faster than
HQC and five orders faster than McEliece keygen.  For signatures,
Falcon and ML-DSA are sub-\SI{2}{\milli\second} while
SPHINCS\textsuperscript{+} signing exceeds \SI{1}{\second}.

\subsection{AEAD Data-Plane Performance}

\begin{table}[htbp]
\centering
\caption{AEAD throughput for 64-byte MAVLink payloads ($n$=200)}
\label{tab:aead-bench}
\footnotesize
\begin{tabular}{@{}l r r r@{}}
\toprule
\textbf{Algorithm} & \textbf{Encrypt (ns)} & \textbf{Decrypt (ns)} & \textbf{Overhead} \\
\midrule
AES-256-GCM        & \num{7877}  & \num{7936}  & 1.00$\times$ \\
ChaCha20-Poly1305  & \num{6741}  & \num{7129}  & 0.88$\times$ \\
Ascon-128a          & \num{4148}  & \num{4241}  & \textbf{0.54$\times$} \\
\bottomrule
\end{tabular}
\end{table}

For 64-byte MAVLink packets, Ascon-128a is 46\% faster than AES-256-GCM
on the ARM Cortex-A76, which lacks AES-NI but has efficient bitsliced
Ascon~paths.  At \SI{320}{\hertz} MAVLink rate, the per-packet AEAD
overhead is $< \SI{8}{\micro\second}$---negligible compared to the
\SI{3.125}{\milli\second} inter-packet interval.

\subsection{End-to-End Suite Handshake Results}

We ran all 72~suites through the full tunnel (MAVProxy $\to$ PQC
Proxy $\to$ encrypted link $\to$ PQC Proxy $\to$ MAVProxy).
71/72 succeeded; one McEliece-460896+SPHINCS\textsuperscript{+}-192s
suite timed out.

\begin{table}[htbp]
\centering
\caption{Suite handshake times by NIST level ($n$=200 each)}
\label{tab:suite-by-level}
\footnotesize
\begin{tabular}{@{}l r r r r r@{}}
\toprule
\textbf{Level} & \textbf{$n$} & \textbf{Mean} & \textbf{Median} & \textbf{P95} & \textbf{Max} \\
 & & \textbf{(ms)} & \textbf{(ms)} & \textbf{(ms)} & \textbf{(ms)} \\
\midrule
L1 & \num{1800} & 880 & 503 & \num{2039} & \num{3742} \\
L3 & \num{1200} & \num{2560} & \num{3178} & \num{4793} & \num{5398} \\
L5 & \num{1600} & \num{9528} & \num{7613} & \num{24339} & \num{36633} \\
\bottomrule
\end{tabular}
\end{table}

\textbf{Regression model} (from 4,600 measurements):
\begin{equation}
\log_{10}(T_\text{hs}) = -1.42 + 0.87 \log_{10}(\text{pk\_size})
  + 0.31 \log_{10}(\text{sig\_size})
\label{eq:regression}
\end{equation}
with $R^2 = 0.96$.  Public-key size is the dominant predictor of
handshake latency.

\subsection{Pareto-Optimal Suites}

We identify the Pareto frontier of security~level vs.~handshake
latency from the single-pass 72-suite run:

\begin{table}[htbp]
\centering
\caption{Pareto-optimal suites (latency vs.\ NIST level)}
\label{tab:pareto}
\footnotesize
\begin{tabular}{@{}l l r r r@{}}
\toprule
\textbf{Suite (KEM+SIG)} & \textbf{NIST} & \textbf{$T_\text{hs}$} & \textbf{Energy} & \textbf{PK} \\
 & & \textbf{(ms)} & \textbf{(mJ)} & \textbf{(B)} \\
\midrule
ML-KEM-512 + Falcon-512 & L1 & 13.1 & $\sim$10 & 800 \\
ML-KEM-768 + ML-DSA-65 & L3 & 14.8 & $\sim$24 & \num{1184} \\
ML-KEM-1024 + Falcon-1024 & L5 & 11.0 & $\sim$24 & \num{1568} \\
\bottomrule
\end{tabular}
\end{table}

All three Pareto-optimal suites use ML-KEM.  No HQC or McEliece suite
appears on the Pareto frontier because their handshake times are
dominated by the KEM keygen/decapsulate operations.

\subsection{System Metrics During Tunnel Operation}

\begin{table}[htbp]
\centering
\caption{Drone system metrics during full-tunnel operation}
\label{tab:system-metrics}
\footnotesize
\begin{tabular}{@{}l r r@{}}
\toprule
\textbf{Metric} & \textbf{Light suite} & \textbf{Heavy suite} \\
 & (ML-KEM-512) & (McE-8192128) \\
\midrule
MAVLink rx rate & \SI{320.3}{\hertz} & \SI{320.3}{\hertz} \\
Heartbeat interval & \SI{999.9}{\milli\second} & \SI{999.9}{\milli\second} \\
Drone CPU (avg) & 25.0\% & 24.8\% \\
Drone CPU (peak) & 45.7\% & 41.1\% \\
Drone temp & \SI{63.8}{\degreeCelsius} & \SI{62.8}{\degreeCelsius} \\
Power (avg) & \SI{3.990}{\watt} & \SI{3.974}{\watt} \\
Power (peak) & \SI{5.102}{\watt} & \SI{4.712}{\watt} \\
Energy total & \SI{435.7}{\joule} & \SI{437.9}{\joule} \\
Energy/handshake & \SI{12.4}{\joule} & \SI{14.4}{\joule} \\
Packet loss & 0.0\% & 0.0\% \\
AEAD encrypt (avg) & \SI{73.3}{\micro\second} & \SI{72.2}{\micro\second} \\
\bottomrule
\end{tabular}
\end{table}

\textbf{Key finding:} Steady-state power and CPU are nearly identical
across suites because the AEAD data plane dominates runtime cost.  The
difference manifests entirely during handshake: heavy suites cause a
brief spike in CPU and power during rekey, but the steady state is
suite-independent.  This validates the policy's focus on
\emph{handshake cost} as the discriminating factor.

% ====================================================================
\section{Graceful Degradation Analysis}
\label{sec:degradation}

\subsection{Rekey Overhead Fraction}

The fraction of time spent in handshake (blackout) for a rekey interval
$R$ is:
\begin{equation}
\Phi(R) = \frac{T_\text{hs}}{R + T_\text{hs}}
\label{eq:overhead}
\end{equation}

\begin{table}[htbp]
\centering
\caption{Rekey overhead $\Phi$ at different intervals (measured $T_\text{hs}$)}
\label{tab:rekey-overhead}
\footnotesize
\begin{tabular}{@{}l r r r r r@{}}
\toprule
\textbf{Suite} & $T_\text{hs}$ & $R$=60s & $R$=300s & $R$=600s & $R$=3600s \\
\midrule
ML-KEM-768 & \SI{4.1}{\milli\second} & 0.007\% & 0.001\% & 0.001\% & ${<}0.001\%$ \\
HQC-256 & \SI{96.3}{\milli\second} & 0.16\% & 0.032\% & 0.016\% & 0.003\% \\
McE-348864 & \SI{287}{\milli\second} & 0.48\% & 0.096\% & 0.048\% & 0.008\% \\
McE-8192128 & \SI{9.2}{\second} & \textbf{13.3\%} & \textbf{2.97\%} & 1.51\% & 0.25\% \\
\bottomrule
\end{tabular}
\end{table}

At $R = \SI{60}{\second}$, ML-KEM suites have negligible overhead
($\Phi < 0.01\%$), while McEliece-8192128 consumes 13.3\% of the
cycle in handshake.  The policy's \texttt{min\_stable\_s = 60}
threshold (proactive rekey, Priority~7) is therefore safe for all
ML-KEM suites but would be catastrophic for McEliece.

\subsection{What We Compromise at Each Security Level}

When the policy downgrades from L5 to L1 under stress, the system
trades:

\begin{table}[htbp]
\centering
\caption{Quantified degradation: L5 $\to$ L3 $\to$ L1 (ML-KEM family)}
\label{tab:degradation}
\footnotesize
\begin{tabular}{@{}l r r r r@{}}
\toprule
\textbf{Metric} & \textbf{L5} & \textbf{L3} & \textbf{L1} & \textbf{L5/L1} \\
\midrule
NIST security & 256-bit & 192-bit & 128-bit & --- \\
Handshake (ms) & 15.3 & 14.8 & 13.1 & 1.2$\times$ \\
KEM keygen ($\mu$s) & 142 & 111 & 116 & 1.2$\times$ \\
KEM encaps ($\mu$s) & 121 & 89 & 66 & 1.8$\times$ \\
PK size (B) & \num{1568} & \num{1184} & 800 & 2.0$\times$ \\
Ciphertext (B) & \num{1568} & \num{1088} & 768 & 2.0$\times$ \\
Energy/hs (mJ) & $\sim$24 & $\sim$24 & $\sim$10 & 2.4$\times$ \\
Rekey $\Phi$@60s & 0.025\% & 0.025\% & 0.022\% & $\sim$1$\times$ \\
\bottomrule
\end{tabular}
\end{table}

\textbf{Insight:} Within the ML-KEM family, the degradation cost is
minimal---L5~$\to$~L1 reduces security from 256-bit to 128-bit quantum
but saves only \SI{2.2}{\milli\second} per handshake and \SI{14}{\milli\joule}
per rekey.  The real benefit of degradation is moving \emph{across KEM
families}: from McEliece/HQC to ML-KEM, which saves \emph{seconds} per
handshake and eliminates blackout-induced MAVLink loss.

\subsection{Cross-Family Degradation}

The policy's tier mapping (\cref{eq:tier}) encodes this insight.
When operating at L5 with McEliece-8192128 (tier~25), the first
downgrade targets L5+HQC-256 (tier~23), then L5+ML-KEM-1024 (tier~20).
This 5-tier drop corresponds to:

\begin{itemize}
  \item Handshake: \SI{9528}{\milli\second}~$\to$~\SI{15}{\milli\second}
    (635$\times$ reduction)
  \item PK size: \SI{1.36}{\mega\byte}~$\to$~\SI{1568}{\byte}
    (867$\times$ reduction)
  \item Rekey $\Phi$@60s: 13.3\%~$\to$~0.025\% (532$\times$ reduction)
  \item \textbf{Security loss: None} (both are NIST L5)
\end{itemize}

This is the policy's most powerful degradation step: replacing
McEliece with ML-KEM at the same NIST level eliminates virtually
all overhead while maintaining the same quantum security guarantee.
The cost is reduced algorithm diversity---both use lattice
assumptions.

\subsection{Energy Budget Under Degradation}

For a 30-minute flight with \SI{3.99}{\watt} baseline power draw
(\SI{7182}{\joule} total budget), the energy consumed by rekeys at
$R = \SI{60}{\second}$ is:

\begin{table}[htbp]
\centering
\caption{Rekey energy over a 30-min flight ($R$=60s, 30 rekeys)}
\label{tab:energy-budget}
\footnotesize
\begin{tabular}{@{}l r r r@{}}
\toprule
\textbf{Suite} & \textbf{E/rekey} & \textbf{30 rekeys} & \textbf{\% budget} \\
\midrule
ML-KEM-512+Falcon-512 & \SI{10}{\milli\joule} & \SI{0.3}{\joule} & 0.004\% \\
ML-KEM-768+ML-DSA-65 & \SI{24}{\milli\joule} & \SI{0.7}{\joule} & 0.010\% \\
McE-348864+Falcon-512 & \SI{12.4}{\joule} & \SI{373}{\joule} & 5.2\% \\
McE-8192128+Falcon-1024 & \SI{14.4}{\joule} & \SI{433}{\joule} & \textbf{6.0\%} \\
\bottomrule
\end{tabular}
\end{table}

ML-KEM rekeys consume $< 0.01\%$ of the flight energy budget.
McEliece rekeys consume up to 6\%, which is significant but not
catastrophic at $R = \SI{60}{\second}$.  The policy's minimum
stable time (\SI{60}{\second}) ensures this is the worst case.

% ====================================================================
\section{Policy Comparison}
\label{sec:comparison}

We compare four scheduling strategies implemented in our system:

\begin{enumerate}
  \item \textbf{LinearLoop}: Round-robin cycling through all suites at
    a fixed interval.  No telemetry awareness.
  \item \textbf{Random}: Randomly selects next suite.  No telemetry
    awareness.
  \item \textbf{DeterministicClock}: Chronos-synchronised 10-second
    rotation through all suites.  Benchmark-oriented.
  \item \textbf{TelemetryAwarePolicyV2}: The adaptive policy described
    in~\cref{sec:policy}.
\end{enumerate}

\begin{table}[htbp]
\centering
\caption{Policy comparison across operational scenarios}
\label{tab:policy-comparison}
\footnotesize
\begin{tabular}{@{}l c c c c@{}}
\toprule
\textbf{Property} & \textbf{Linear} & \textbf{Random} & \textbf{Clock} & \textbf{Adaptive} \\
\midrule
Telemetry-aware & \texttimes & \texttimes & \texttimes & \checkmark \\
Battery-aware & \texttimes & \texttimes & \texttimes & \checkmark \\
Thermal-aware & \texttimes & \texttimes & \texttimes & \checkmark \\
Link-quality-aware & \texttimes & \texttimes & \texttimes & \checkmark \\
Avoids heavy KEM\textsuperscript{a} & \texttimes & \texttimes & \texttimes & \checkmark \\
Blackout bounded & \texttimes & \texttimes & \texttimes & \checkmark \\
Oscillation-free & \checkmark & \texttimes & \checkmark & \checkmark \\
Deterministic & \checkmark & \texttimes & \checkmark & \checkmark \\
Suite diversity & All & All & All & Filtered \\
Use case & Benchmark & Test & Benchmark & \textbf{Flight} \\
\bottomrule
\multicolumn{5}{@{}l}{\textsuperscript{a}Avoids McEliece/SPHINCS\textsuperscript{+} during stress.}
\end{tabular}
\end{table}

\subsection{Failure Scenario Analysis}

Consider a scenario where the drone is armed, flying, with battery at
\SI{15}{\volt} (warn level) and temperature at \SI{72}{\degreeCelsius}
(above warn threshold):

\begin{itemize}
  \item \textbf{LinearLoop/Clock}: Will cycle to McEliece-8192128,
    causing a \SI{9.2}{\second} handshake blackout.  During this
    blackout, the GCS receives no heartbeats and may trigger a
    failsafe.  CPU spikes during keygen may push temperature above
    \SI{80}{\degreeCelsius}, causing thermal throttling.

  \item \textbf{Random}: Has a $1/72 \approx 1.4\%$ chance per
    selection of hitting the heaviest suite.  Over 30~rekeys in a
    flight, the probability of at least one McEliece-8192128
    selection is $1 - (71/72)^{30} \approx 34\%$.

  \item \textbf{Adaptive}: Priority~6 (Stress) fires because
    $T > \SI{70}{\degreeCelsius}$ (warn threshold).  The policy
    DOWNGRADEs to the next lighter suite.  If already on the
    lightest ML-KEM suite, it HOLDs.  McEliece and SPHINCS\textsuperscript{+}
    suites are never reached because the filter
    (\texttt{allowed\_aead}, \texttt{max\_nist\_level}) and tier
    ordering place them at the end.
\end{itemize}

\subsection{Benchmark Mode vs.\ Flight Mode}

The two modes serve complementary purposes:

\begin{itemize}
  \item \textbf{Deterministic mode} (BenchmarkPolicy): Fixed
    \SI{10}{\second} intervals, sequential cycling through all
    72~suites.  Designed to collect comprehensive performance data
    without human intervention.  Total run time:
    $72 \times 110\text{s} \approx 2\text{h}12\text{m}$.

  \item \textbf{Intelligent mode} (TelemetryAwarePolicyV2): Filters
    suites by \texttt{allowed\_aead} and \texttt{max\_nist\_level},
    starts at the lightest tier, and adapts based on real-time
    telemetry.  Designed for actual flight operations.
\end{itemize}

% ====================================================================
\section{Related Work}
\label{sec:related}

Post-quantum key exchange has been benchmarked extensively on x86
platforms~\cite{nist-pqc} and to a lesser extent on ARM
Cortex-M~\cite{pqm4} and Cortex-A~\cite{pqcrypto-arm}.  However,
prior work focuses on \emph{static} algorithm selection, not
\emph{runtime adaptive} suite switching.

MAVLink security has been studied through MAVSec~\cite{mavsec} and
protocol-level encryption proposals, but none integrate PQC algorithms
or address the rekey scheduling problem.

TLS~1.3 post-quantum integration (e.g., Cloudflare/Google
experiments~\cite{pq-tls}) addresses key exchange but operates in a
client--server model with ample compute resources, unlike the
constrained UAV scenario.

To our knowledge, this is the first system that:
\begin{enumerate}
  \item Implements runtime PQC suite switching on a drone.
  \item Uses real-time telemetry to drive suite selection.
  \item Quantifies the degradation trade-off across 72~suites on
    ARM hardware.
\end{enumerate}

% ====================================================================
\section{Conclusion}
\label{sec:conclusion}

We presented a telemetry-aware adaptive rekey policy for PQC-secured
UAV tunnels, backed by \num{19600}~benchmarked cryptographic operations
and 71~end-to-end tunnel runs on a Raspberry~Pi~5.

Our key findings are:

\begin{enumerate}
  \item \textbf{ML-KEM is the only viable KEM family for frequent
    rekeying.}  Its sub-\SI{15}{\milli\second} handshakes yield
    $\Phi < 0.03\%$ overhead at $R = \SI{60}{\second}$.  McEliece
    and HQC are orders of magnitude slower.

  \item \textbf{Graceful degradation within ML-KEM is nearly free.}
    Downgrading from L5 to L1 saves \SI{2.2}{\milli\second} per
    handshake---the real gain comes from cross-family degradation
    (McEliece $\to$ ML-KEM), which saves seconds.

  \item \textbf{AEAD choice is irrelevant for the rekey decision.}
    All three AEADs differ by $< \SI{4}{\micro\second}$ per packet
    on the steady-state data plane; the KEM dominates handshake cost.

  \item \textbf{The adaptive policy prevents thermal and link
    failures} that would occur under naive round-robin or random
    scheduling by filtering heavy suites and enforcing hysteresis.

  \item \textbf{The recommended production suite} is ML-KEM-768 +
    ML-DSA-65 + AES-256-GCM (NIST L3), with the policy starting at
    this tier and degrading to ML-KEM-512 + Falcon-512 (L1) under
    stress.
\end{enumerate}

\subsection{Future Work}

\begin{itemize}
  \item Real flight testing with the adaptive policy active.
  \item Machine-learning extension that learns optimal thresholds
    from flight logs.
  \item Hybrid PQC + classical key exchange for defence-in-depth.
  \item TLS-based authenticated control channel (currently plaintext
    JSON-over-TCP).
  \item Session resumption to avoid full handshakes on rekey.
\end{itemize}

% ====================================================================
% References
% ====================================================================
\begin{thebibliography}{10}

\bibitem{nist-pqc}
National Institute of Standards and Technology,
``Post-Quantum Cryptography Standardization,''
NIST FIPS 203/204/205, 2024.

\bibitem{mavlink}
MAVLink Developer Guide,
``MAVLink 2.0 Protocol,''
\url{https://mavlink.io/en/}, 2024.

\bibitem{pqm4}
M.~J. Kannwischer, J.~Rijneveld, P.~Schwabe, and K.~Stoffelen,
``pqm4: Testing and Benchmarking NIST PQC on ARM Cortex-M4,''
IACR ePrint 2019/844.

\bibitem{pqcrypto-arm}
W.~Cheng \emph{et al.},
``Post-Quantum Cryptography on ARM Cortex-A: Benchmarks and Analysis,''
IEEE Access, vol.~10, 2022.

\bibitem{mavsec}
N.~Shoufan, H.~El-Hajj, and S.~Kunz,
``MAVSec: Securing the MAVLink Protocol for Unmanned Aerial Systems,''
Proc. IEEE MILCOM, 2019.

\bibitem{pq-tls}
K.~Kwiatkowski and N.~Sullivan,
``Measuring TLS Key Exchange with Post-Quantum KEM,''
Proc. NDSS Workshop on Measurements, 2020.

\bibitem{liboqs}
D.~Stebila and M.~Mosca,
``Post-Quantum Key Exchange for the Internet and the Open Quantum Safe Project,''
SAC 2016, LNCS 10532, pp.~14--37.

\bibitem{falcon}
P.-A. Fouque \emph{et al.},
``FALCON: Fast-Fourier Lattice-based Compact Signatures over NTRU,''
NIST PQC Round 3 Submission, 2022.

\bibitem{sphincs}
D.~J. Bernstein \emph{et al.},
``SPHINCS\textsuperscript{+}: Submission to the NIST PQC Standardization Process,''
2022.

\bibitem{hqc}
C.~Aguilar Melchor \emph{et al.},
``HQC: Hamming Quasi-Cyclic,''
NIST PQC Round 4 Submission, 2023.

\end{thebibliography}

\balance
\end{document}
